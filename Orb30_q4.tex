\documentclass{article}
% a4paper

%\usepackage[dvips]{epsfig}
%\usepackage{cours11, cours}
%\usepackage{fancyheadings}
%\usepackage{calc}
\usepackage{amsmath}
\usepackage{amssymb}
\usepackage{latexsym}
\usepackage{epsfig}
\usepackage{enumerate}
%\usepackage{supertabular}
%\usepackage{wrapfig}
%\usepackage{blackbrd}
%\usepackage{epic,eepic}
\usepackage{rotating}
\usepackage{multicol}
%\usepackage{multirow}
\usepackage{makeidx} % additional command see
\usepackage{rotating}
\usepackage{array}
\usepackage{tikz}
\usepackage{longtable}
\usepackage{anyfontsize}
\usepackage{t1enc}
%\usepackage{amsmath,amsfonts}


%\usepackage[mtbold,mtplusscr]{mathtime}
% lucidacal,lucidascr,

%\usepackage{mathtimy}
%\usepackage{bm}
%\usepackage{avant}
%\usepackage{basker}
%\usepackage{bembo}
%\usepackage{bookman}
%\usepackage{chancery}
%\usepackage{garamond}
%\usepackage{helvet}
%\usepackage{newcent}
%\usepackage{palatino}
%\usepackage{times}
%\usepackage{pifont}
\usepackage{fullpage}
\usepackage[top=1in,bottom=1in,right=1in,left=1in]{geometry}



%\parindent=0pt


%\topmargin=0pt
%\headsep=18pt
%\footskip=45pt
%\mathsurround=1pt
%\evensidemargin=0pt
%\oddsidemargin=15pt

%\setlength{\textheight}{\baselineskip*41+\topskip}

\newcommand{\sectionline}{
   \nointerlineskip \vspace{\baselineskip}
   \hspace{\fill}\rule{0.9\linewidth}{1.7pt}\hspace{\fill}
   \par\nointerlineskip \vspace{\baselineskip}
   }
\newcommand\setTBstruts{\def\T{\rule{0pt}{2.6ex}}%
\def\B{\rule[-1.2ex]{0pt}{0pt}}}
\newcommand{\ans}[1]{\\{\bf ANSWER}: {#1}}
\newcommand{\Aut}{{\rm Aut}}
\newcommand{\Sym}{{\rm Sym}}
\newcommand{\sFix}{{\cal Fix}}
\newcommand{\sOrbits}{{\cal Orbits}}
\newcommand{\Stab}{{\rm Stab}}
\newcommand{\Fix}{{\rm Fix}}
\newcommand{\fix}{{\rm fix}}
\newcommand{\Orbits}{{\rm Orbits}}
\newcommand{\PG}{{\rm PG}}
\newcommand{\AG}{{\rm AG}}
\newcommand{\SQS}{{\rm SQS}}
\newcommand{\STS}{{\rm STS}}
\newcommand{\PSL}{{\rm PSL}}
\newcommand{\PGL}{{\rm PGL}}
\newcommand{\PSSL}{{\rm P\Sigma L}}
\newcommand{\PGGL}{{\rm P\Gamma L}}
\newcommand{\SL}{{\rm SL}}
\newcommand{\GL}{{\rm GL}}
\newcommand{\SSL}{{\rm \Sigma L}}
\newcommand{\GGL}{{\rm \Gamma L}}
\newcommand{\ASL}{{\rm ASL}}
\newcommand{\AGL}{{\rm AGL}}
\newcommand{\ASSL}{{\rm A\Sigma L}}
\newcommand{\AGGL}{{\rm A\Gamma L}}
\newcommand{\PSU}{{\rm PSU}}
\newcommand{\HS}{{\rm HS}}
\newcommand{\Hol}{{\rm Hol}}
\newcommand{\SO}{{\rm SO}}
\newcommand{\ASO}{{\rm ASO}}
\newcommand{\la}{\langle}
\newcommand{\ra}{\rangle}
\newcommand{\cA}{{\cal A}}
\newcommand{\cB}{{\cal B}}
\newcommand{\cC}{{\cal C}}
\newcommand{\cD}{{\cal D}}
\newcommand{\cE}{{\cal E}}
\newcommand{\cF}{{\cal F}}
\newcommand{\cG}{{\cal G}}
\newcommand{\cH}{{\cal H}}
\newcommand{\cI}{{\cal I}}
\newcommand{\cJ}{{\cal J}}
\newcommand{\cK}{{\cal K}}
\newcommand{\cL}{{\cal L}}
\newcommand{\cM}{{\cal M}}
\newcommand{\cN}{{\cal N}}
\newcommand{\cO}{{\cal O}}
\newcommand{\cP}{{\cal P}}
\newcommand{\cQ}{{\cal Q}}
\newcommand{\cR}{{\cal R}}
\newcommand{\cS}{{\cal S}}
\newcommand{\cT}{{\cal T}}
\newcommand{\cU}{{\cal U}}
\newcommand{\cV}{{\cal V}}
\newcommand{\cW}{{\cal W}}
\newcommand{\cX}{{\cal X}}
\newcommand{\cY}{{\cal Y}}
\newcommand{\cZ}{{\cal Z}}
\newcommand{\rmA}{{\rm A}}
\newcommand{\rmB}{{\rm B}}
\newcommand{\rmC}{{\rm C}}
\newcommand{\rmD}{{\rm D}}
\newcommand{\rmE}{{\rm E}}
\newcommand{\rmF}{{\rm F}}
\newcommand{\rmG}{{\rm G}}
\newcommand{\rmH}{{\rm H}}
\newcommand{\rmI}{{\rm I}}
\newcommand{\rmJ}{{\rm J}}
\newcommand{\rmK}{{\rm K}}
\newcommand{\rmL}{{\rm L}}
\newcommand{\rmM}{{\rm M}}
\newcommand{\rmN}{{\rm N}}
\newcommand{\rmO}{{\rm O}}
\newcommand{\rmP}{{\rm P}}
\newcommand{\rmQ}{{\rm Q}}
\newcommand{\rmR}{{\rm R}}
\newcommand{\rmS}{{\rm S}}
\newcommand{\rmT}{{\rm T}}
\newcommand{\rmU}{{\rm U}}
\newcommand{\rmV}{{\rm V}}
\newcommand{\rmW}{{\rm W}}
\newcommand{\rmX}{{\rm X}}
\newcommand{\rmY}{{\rm Y}}
\newcommand{\rmZ}{{\rm Z}}
\newcommand{\bA}{{\bf A}}
\newcommand{\bB}{{\bf B}}
\newcommand{\bC}{{\bf C}}
\newcommand{\bD}{{\bf D}}
\newcommand{\bE}{{\bf E}}
\newcommand{\bF}{{\bf F}}
\newcommand{\bG}{{\bf G}}
\newcommand{\bH}{{\bf H}}
\newcommand{\bI}{{\bf I}}
\newcommand{\bJ}{{\bf J}}
\newcommand{\bK}{{\bf K}}
\newcommand{\bL}{{\bf L}}
\newcommand{\bM}{{\bf M}}
\newcommand{\bN}{{\bf N}}
\newcommand{\bO}{{\bf O}}
\newcommand{\bP}{{\bf P}}
\newcommand{\bQ}{{\bf Q}}
\newcommand{\bR}{{\bf R}}
\newcommand{\bS}{{\bf S}}
\newcommand{\bT}{{\bf T}}
\newcommand{\bU}{{\bf U}}
\newcommand{\bV}{{\bf V}}
\newcommand{\bW}{{\bf W}}
\newcommand{\bX}{{\bf X}}
\newcommand{\bY}{{\bf Y}}
\newcommand{\bZ}{{\bf Z}}
\newcommand{\sA}{{\cal A}}
\newcommand{\sB}{{\cal B}}
\newcommand{\sC}{{\cal C}}
\newcommand{\sD}{{\cal D}}
\newcommand{\sE}{{\cal E}}
\newcommand{\sF}{{\cal F}}
\newcommand{\sG}{{\cal G}}
\newcommand{\sH}{{\cal H}}
\newcommand{\sI}{{\cal I}}
\newcommand{\sJ}{{\cal J}}
\newcommand{\sK}{{\cal K}}
\newcommand{\sL}{{\cal L}}
\newcommand{\sM}{{\cal M}}
\newcommand{\sN}{{\cal N}}
\newcommand{\sO}{{\cal O}}
\newcommand{\sP}{{\cal P}}
\newcommand{\sQ}{{\cal Q}}
\newcommand{\sR}{{\cal R}}
\newcommand{\sS}{{\cal S}}
\newcommand{\sT}{{\cal T}}
\newcommand{\sU}{{\cal U}}
\newcommand{\sV}{{\cal V}}
\newcommand{\sW}{{\cal W}}
\newcommand{\sX}{{\cal X}}
\newcommand{\sY}{{\cal Y}}
\newcommand{\sZ}{{\cal Z}}
\newcommand{\frakA}{{\mathfrak A}}
\newcommand{\frakB}{{\mathfrak B}}
\newcommand{\frakC}{{\mathfrak C}}
\newcommand{\frakD}{{\mathfrak D}}
\newcommand{\frakE}{{\mathfrak E}}
\newcommand{\frakF}{{\mathfrak F}}
\newcommand{\frakG}{{\mathfrak G}}
\newcommand{\frakH}{{\mathfrak H}}
\newcommand{\frakI}{{\mathfrak I}}
\newcommand{\frakJ}{{\mathfrak J}}
\newcommand{\frakK}{{\mathfrak K}}
\newcommand{\frakL}{{\mathfrak L}}
\newcommand{\frakM}{{\mathfrak M}}
\newcommand{\frakN}{{\mathfrak N}}
\newcommand{\frakO}{{\mathfrak O}}
\newcommand{\frakP}{{\mathfrak P}}
\newcommand{\frakQ}{{\mathfrak Q}}
\newcommand{\frakR}{{\mathfrak R}}
\newcommand{\frakS}{{\mathfrak S}}
\newcommand{\frakT}{{\mathfrak T}}
\newcommand{\frakU}{{\mathfrak U}}
\newcommand{\frakV}{{\mathfrak V}}
\newcommand{\frakW}{{\mathfrak W}}
\newcommand{\frakX}{{\mathfrak X}}
\newcommand{\frakY}{{\mathfrak Y}}
\newcommand{\frakZ}{{\mathfrak Z}}
\newcommand{\fraka}{{\mathfrak a}}
\newcommand{\frakb}{{\mathfrak b}}
\newcommand{\frakc}{{\mathfrak c}}
\newcommand{\frakd}{{\mathfrak d}}
\newcommand{\frake}{{\mathfrak e}}
\newcommand{\frakf}{{\mathfrak f}}
\newcommand{\frakg}{{\mathfrak g}}
\newcommand{\frakh}{{\mathfrak h}}
\newcommand{\fraki}{{\mathfrak i}}
\newcommand{\frakj}{{\mathfrak j}}
\newcommand{\frakk}{{\mathfrak k}}
\newcommand{\frakl}{{\mathfrak l}}
\newcommand{\frakm}{{\mathfrak m}}
\newcommand{\frakn}{{\mathfrak n}}
\newcommand{\frako}{{\mathfrak o}}
\newcommand{\frakp}{{\mathfrak p}}
\newcommand{\frakq}{{\mathfrak q}}
\newcommand{\frakr}{{\mathfrak r}}
\newcommand{\fraks}{{\mathfrak s}}
\newcommand{\frakt}{{\mathfrak t}}
\newcommand{\fraku}{{\mathfrak u}}
\newcommand{\frakv}{{\mathfrak v}}
\newcommand{\frakw}{{\mathfrak w}}
\newcommand{\frakx}{{\mathfrak x}}
\newcommand{\fraky}{{\mathfrak y}}
\newcommand{\frakz}{{\mathfrak z}}
\newcommand{\Tetra}{{\mathfrak Tetra}}
\newcommand{\Cube}{{\mathfrak Cube}}
\newcommand{\Octa}{{\mathfrak Octa}}
\newcommand{\Dode}{{\mathfrak Dode}}
\newcommand{\Ico}{{\mathfrak Ico}}
\newcommand{\bbF}{{\mathbb F}}
\newcommand{\bbQ}{{\mathbb Q}}
\newcommand{\bbC}{{\mathbb C}}
\newcommand{\bbR}{{\mathbb R}}



%\makeindex

\begin{document} 
\setTBstruts

\bibliographystyle{plain}
%\large

{\allowdisplaybreaks%




%\makeindex

%\renewcommand{\labelenumi}{(\roman{enumi})}

\title{Rank-65542 over GF(4)}
\author{}%end author
%\date{}
\maketitle%
\pagenumbering{roman}
%\thispagestyle{empty}
%\input et.tex%
%\thispagestyle{empty}%\phantom{page2}%\clearpage%
%\addcontentsline{toc}{chapter}{Inhaltsverzeichnis}%
%\tableofcontents
%\listofsymbols
\pagenumbering{arabic}
%\pagenumbering{roman}



\subsection*{General information}
Points on lines:
$$
5^4$$
Lines on points:
$$
4,\,1^{16}$$
\subsubsection*{Singular Points}
The surface has 5 singular points:\\
\begin{align*}
S_{0} &= P_{2}=\bP(0, 0, 1, 0) = \bP(0, 0, 1, 0)\\
S_{1} &= P_{3}=\bP(0, 0, 0, 1) = \bP(0, 0, 0, 1)\\
S_{2} &= P_{38}=\bP(0, 0, 1, 1) = \bP(0, 0, 1, 1)\\
S_{3} &= P_{53}=\bP(0, 0, \omega , 1) = \bP(0, 0, 2, 1)\\
S_{4} &= P_{69}=\bP(0, 0, \omega^{2}, 1) = \bP(0, 0, 3, 1)\\
\end{align*}
\subsection*{The 4 Lines}
$$
\ell_{0} = 
\left[
\begin{array}{*{4}{r}}
0 & 0 & 1 & 0\\
0 & 0 & 0 & 1\\
\end{array}
\right]
_{356}
=
\left[
\begin{array}{*{4}c}
0  & 0  & 1  & 0\\
0  & 0  & 0  & 1\\
\end{array}
\right]_{356}
={\rm\bf Pl}(0,1,0,0,0,0 )_{1}$$
$$
\ell_{1} = 
\left[
\begin{array}{*{4}{r}}
1 & 1 & 0 & 0\\
0 & 0 & 0 & 1\\
\end{array}
\right]
_{41}
=
\left[
\begin{array}{*{4}c}
1  & 1  & 0  & 0\\
0  & 0  & 0  & 1\\
\end{array}
\right]_{41}
={\rm\bf Pl}(0,0,0,1,1,0 )_{53}$$
$$
\ell_{2} = 
\left[
\begin{array}{*{4}{r}}
1 & \omega^{2} & 0 & 0\\
0 & 0 & 0 & 1\\
\end{array}
\right]
_{83}
=
\left[
\begin{array}{*{4}c}
1  & 3  & 0  & 0\\
0  & 0  & 0  & 1\\
\end{array}
\right]_{83}
={\rm\bf Pl}(0,0,0,3,1,0 )_{67}$$
$$
\ell_{3} = 
\left[
\begin{array}{*{4}{r}}
1 & \omega  & 0 & 0\\
0 & 0 & 0 & 1\\
\end{array}
\right]
_{62}
=
\left[
\begin{array}{*{4}c}
1  & 2  & 0  & 0\\
0  & 0  & 0  & 1\\
\end{array}
\right]_{62}
={\rm\bf Pl}(0,0,0,2,1,0 )_{60}$$
Rank of lines: ( 356, 41, 83, 62 )\\
Rank of points on Klein quadric: ( 1, 53, 67, 60 )\\
\subsubsection*{Eckardt Points}
The surface has 0 Eckardt points:\\
\subsubsection*{Double Points}
The surface has 0 Double points:\\
The double points on the surface are:\\
\begin{multicols}{2}
\noindent
\end{multicols}
\subsubsection*{Single Points}
The surface has 16 single points:\\
The single points on the surface are:\\
\begin{multicols}{2}
\noindent
0 : $P_{2}=( 0, 0, 1, 0 )$ lies on line $\ell_{0}$\\
1 : $P_{5}=( 1, 1, 0, 0 )$ lies on line $\ell_{1}$\\
2 : $P_{6}=( 2, 1, 0, 0 )$ lies on line $\ell_{2}$\\
3 : $P_{7}=( 3, 1, 0, 0 )$ lies on line $\ell_{3}$\\
4 : $P_{27}=( 1, 1, 0, 1 )$ lies on line $\ell_{1}$\\
5 : $P_{28}=( 2, 1, 0, 1 )$ lies on line $\ell_{2}$\\
6 : $P_{29}=( 3, 1, 0, 1 )$ lies on line $\ell_{3}$\\
7 : $P_{31}=( 1, 2, 0, 1 )$ lies on line $\ell_{3}$\\
8 : $P_{32}=( 2, 2, 0, 1 )$ lies on line $\ell_{1}$\\
9 : $P_{33}=( 3, 2, 0, 1 )$ lies on line $\ell_{2}$\\
10 : $P_{35}=( 1, 3, 0, 1 )$ lies on line $\ell_{2}$\\
11 : $P_{36}=( 2, 3, 0, 1 )$ lies on line $\ell_{3}$\\
12 : $P_{37}=( 3, 3, 0, 1 )$ lies on line $\ell_{1}$\\
13 : $P_{38}=( 0, 0, 1, 1 )$ lies on line $\ell_{0}$\\
14 : $P_{53}=( 0, 0, 2, 1 )$ lies on line $\ell_{0}$\\
15 : $P_{69}=( 0, 0, 3, 1 )$ lies on line $\ell_{0}$\\
\end{multicols}
The single points on the surface are:\\
\subsubsection*{Points on surface but on no line}
The surface has 0 points not on any line:\\
$$
\begin{array}{r|*{10}{r}}
 & 0 & 1 & 2 & 3 & 4 & 5 & 6 & 7 & 8 & 9\\
\hline
\end{array}
$$
The points on the surface but not on lines are:\\
\begin{multicols}{2}
\noindent
\end{multicols}
\subsection*{Line Intersection Graph}
{\arraycolsep=1pt
$$
\begin{array}{rr|*{4}r}
 &  & 0 & 1 & 2 & 3\\
\hline
0 &  & 0 & 1 & 1 & 1\\
1 &  & 1 & 0 & 1 & 1\\
2 &  & 1 & 1 & 0 & 1\\
3 &  & 1 & 1 & 1 & 0\\
\end{array}
$$
}%%
Neighbor sets in the line intersection graph:\\
Line 0 intersects 
$$
\begin{array}{|r*{3}{|c}|}
\hline
\mbox{Line}  & \ell_{1} & \ell_{2} & \ell_{3}\\
\hline
\mbox{in point}  & P_{3} & P_{3} & P_{3}\\
\hline
\end{array}
$$
Line 1 intersects 
$$
\begin{array}{|r*{3}{|c}|}
\hline
\mbox{Line}  & \ell_{0} & \ell_{2} & \ell_{3}\\
\hline
\mbox{in point}  & P_{3} & P_{3} & P_{3}\\
\hline
\end{array}
$$
Line 2 intersects 
$$
\begin{array}{|r*{3}{|c}|}
\hline
\mbox{Line}  & \ell_{0} & \ell_{1} & \ell_{3}\\
\hline
\mbox{in point}  & P_{3} & P_{3} & P_{3}\\
\hline
\end{array}
$$
Line 3 intersects 
$$
\begin{array}{|r*{3}{|c}|}
\hline
\mbox{Line}  & \ell_{0} & \ell_{1} & \ell_{2}\\
\hline
\mbox{in point}  & P_{3} & P_{3} & P_{3}\\
\hline
\end{array}
$$
The surface has 17 points:\\
The points on the surface are:\\
\begin{multicols}{3}
\noindent
0 : $P_{2}=( 0, 0, 1, 0 )$\\
1 : $P_{3}=( 0, 0, 0, 1 )$\\
2 : $P_{5}=( 1, 1, 0, 0 )$\\
3 : $P_{6}=( 2, 1, 0, 0 )$\\
4 : $P_{7}=( 3, 1, 0, 0 )$\\
5 : $P_{27}=( 1, 1, 0, 1 )$\\
6 : $P_{28}=( 2, 1, 0, 1 )$\\
7 : $P_{29}=( 3, 1, 0, 1 )$\\
8 : $P_{31}=( 1, 2, 0, 1 )$\\
9 : $P_{32}=( 2, 2, 0, 1 )$\\
10 : $P_{33}=( 3, 2, 0, 1 )$\\
11 : $P_{35}=( 1, 3, 0, 1 )$\\
12 : $P_{36}=( 2, 3, 0, 1 )$\\
13 : $P_{37}=( 3, 3, 0, 1 )$\\
14 : $P_{38}=( 0, 0, 1, 1 )$\\
15 : $P_{53}=( 0, 0, 2, 1 )$\\
16 : $P_{69}=( 0, 0, 3, 1 )$\\
\end{multicols}


%\bibliographystyle{gerplain}% wird oben eingestellt
%\addcontentsline{toc}{section}{References}
%\bibliography{../MY_BIBLIOGRAPHY/anton}
% ACHTUNG: nicht vergessen:
% die Zeile
%\addcontentsline{toc}{chapter}{Literaturverzeichnis}
% muss per Hand in d.bbl eingefuegt werden !
% nach \begin{thebibliography}{100}

%\begin{theindex}

%\clearpage
%\addcontentsline{toc}{chapter}{Index}
%\input{apd.ind}

%\printindex
%\end{theindex}

}% allowdisplaybreaks

\end{document}


